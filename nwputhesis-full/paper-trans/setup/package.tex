%%%%%%%%%%%%%%%%%%%%%%%%%%%%%%%%%%%%%%%%%%%%%%%%%%%%%%%%%%%%%%%%%%%%%%%%%
%
%   LaTeX File for Doctor (Master) Thesis of Tsinghua University
%   LaTeX + CJK     清华大学博士(硕士)论文模板
%   Based on Wang Tianshu's Template for XJTU
%	Version: 1.00
%   Last Update: 2003-09-12
%
%%%%%%%%%%%%%%%%%%%%%%%%%%%%%%%%%%%%%%%%%%%%%%%%%%%%%%%%%%%%%%%%%%%%%%%%%
%   Copyright 2002-2003  by  Lei Wang (BaconChina)       (bcpub@sina.com)
%%%%%%%%%%%%%%%%%%%%%%%%%%%%%%%%%%%%%%%%%%%%%%%%%%%%%%%%%%%%%%%%%%%%%%%%%

%%%%%%%%%%%%%%%%%%%%%%%%%%%%%%%%%%%%%%%%%%%%%%%%%%%%%%%%%%%%%%%%%%%%%%%%%
%
%   LaTeX File for phd thesis of xi'an Jiao Tong University
%
%%%%%%%%%%%%%%%%%%%%%%%%%%%%%%%%%%%%%%%%%%%%%%%%%%%%%%%%%%%%%%%%%%%%%%%%%
%   Copyright 2002  by  Wang Tianshu    (tswang@asia.com)
%%%%%%%%%%%%%%%%%%%%%%%%%%%%%%%%%%%%%%%%%%%%%%%%%%%%%%%%%%%%%%%%%%%%%%%%%

%%%%%%%%%%%%%%%%%%%%%%%%%%%%%%%%%%%%%%%%%%%%%%%%%%%%%%%%%%%
%
% 引用的宏包和相应的定义
%
%%%%%%%%%%%%%%%%%%%%%%%%%%%%%%%%%%%%%%%%%%%%%%%%%%%%%%%%%%%

\usepackage[dvips]{graphicx}
\usepackage{subfigure}
% 支持彩色
\usepackage{color}
% eps图像
\usepackage{epsfig}

%\else
%\usepackage[dvips]{graphicx}
%\usepackage{subfigure}
%\fi

% 首行缩进宏包
\usepackage{indentfirst}

% 版面控制宏包,定义规定的版面尺寸
\usepackage[%top=2cm,
            %body={14.6true cm,22true cm},
            twosideshift=0 pt,
            %headheight=1.0true cm
		top=1in,bottom=1.5in,left=1.25in,right=1.25in
            ]{geometry}

% 脚注控制
\usepackage[perpage,symbol]{footmisc}

% AMSLaTeX宏包 用来排出更加漂亮的公式
\usepackage{amsmath}
\usepackage{amssymb}

% 不同于\mathcal or \mathfrak 之类的英文花体字体
\usepackage{mathrsfs}

% 定理类环境宏包,其中 amsmath 选项用来兼容 AMS LaTeX 的宏包
\usepackage[amsmath,thmmarks]{ntheorem}

% 因为图形可浮动到当前页的顶部,所以它可能会出现
% 在它所在文本的前面. 要防止这种情况,可使用 flafter
% 宏包
%\usepackage{flafter}

%浮动图形控制宏包
%允许上一个section的浮动图形出现在下一个section的开始部分
%该宏包提供处理浮动对象的 \FloatBarrier 命令,使所有未处
%理的浮动图形立即被处理
\usepackage[below]{placeins}

% 图文混排用宏包
%\usepackage{floatflt}

% 图形和表格的控制
\usepackage{rotating}

% tex1cm宏包,控制字体的大小
\usepackage{type1cm}

% 控制标题的宏包
\usepackage[sf]{titlesec}

% 控制目录的宏包
\usepackage{titletoc}

% 处理数学公式中的黑斜体的宏包
\usepackage{bm}

%可将浮动对象放置到文件的最后
%\usepackage{endfloat}

% fancyhdr宏包 页眉和页脚的相关定义
\usepackage{fancyhdr}
\usepackage{fancyref}

% 支持引用的宏包
\usepackage{cite}

%浮动图形和表格标题样式
\usepackage{caption2}

% 定制表格和图形的多行标题行距
\usepackage{setspace}

% 打印当前页面格式的宏包
\usepackage{layouts}

% 使用Times字体的宏包
%\usepackage{times}

% qiuying add
\usepackage[CJKtextspaces]{xeCJK}
\usepackage{tikz}

% 生成带书签的pdf
\usepackage[dvipdfmx,
            CJKbookmarks=true,
            bookmarksnumbered=true,
            bookmarksopen=true,
            pdfborder=001,
            citecolor=blue,
            anchorcolor=green,
            urlcolor=blue,
	  pdfcreator={XeTeX,XeCJK},
	  pdfproducer={XeTeX},% 这个好像没起作用? 
	  pdftitle={四位无线链路估计},
	  pdfsubject={4bitle},
	  pdfauthor={qiuying},
	  pdfkeywords={4bitle},
	  %pdfpagemode={FullScreen},
	  colorlinks={true},
	  linkcolor={blue} %改变目录、引用的颜色
            ]{hyperref}
